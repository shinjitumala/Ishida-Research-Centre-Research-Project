\documentclass[a4j,11pt]{jarticle}
\usepackage{epsfig,here}
\usepackage{url}

\usepackage{color}
\usepackage{listings}

\usepackage{amsmath}

\setlength{\textwidth}{1.1\textwidth}
\setlength{\oddsidemargin}{-3pt}
\setlength{\evensidemargin}{\oddsidemargin}
\setlength{\topmargin}{10mm}
\setlength{\headheight}{0mm}
\setlength{\headsep}{0mm}

\begin{document}

\begin{center}
%\noindent
 \vspace{10mm}

{\bf {\huge 3Q 研究プロジェクト(石田研)}}
%\end{center}

\vspace{80mm}

提出日:2018年10月31日

\vspace{10mm}

系/学科/類:情報工学系

\vspace{10mm}

学籍番号:16B13354

\vspace{20mm}

{\bf {\LARGE 氏名:星野 シンジ}}
\end{center}

\newpage

\section{ハイパーパラメーターの調整}

・momentum

エラー関数の大域的最小値を探そうとする際に、局所解を求めるようになっている。
この値が大きければ大きいほど、アルゴリズムの「勢い」のようなものが大きくなり、
ある局所解が求まったとしても、より最適な解を探そうとするようになる。

0.9から1.5の間あたりでスコアが小さくなった。2程度まではスコアが上がり続けたが、上げすぎると安定点が見つからなくなる。


・batch\_size

一度のイテレーションに用いられるトレーニングサンプルの数。

小さすぎると、結果にノイズが乗りやすくなる。一方で、大きすぎると、テストデータが足りなくなる。
200程度がこの場合最適らしい。


・learning\_rate

アルゴリズムがデータから学習する際に、新しいデータの取り込みに対する
感度を示す。値が大きければ大きいほど、新しいデータに強く適応するようになる。

1e-3周辺を谷として、1e-2と1e-6あたりに山があるようだ、ただしこれはdecayが1e-6の場合である。


・decay

この値が大きければ大きいほど、データを取り込むにつれてlearning\_rateが小さくなる
速さが大きくなる。

1e-7程度で最も大きくなる。大きくすると、1e-6の時よりも少し上がるが、1e-7の時以上には上がらない。

・nb\_epoch

トレーニングサンプル間で、何度epochを行うか。epochとは、すべてのトレーニングデータ間で
フォワードパスとバックワードパスを一度行うこと。つまり、各データに対してイテレーションを
行うことである。

2回程度が最適だということが分かった。


・dropout
---
overfittingを避けるために、ノード間のウェイトを均等にする必要がある。そのために、
dropoutというものが行われるが、これはそれが起きる頻度を示す。

0.1程度で最大になっている。大きすぎたり小さすぎたりすると、大幅にスコアが下がった。


・nb\_layers, layer\_sizes

ニューラルネットワークの層の数、および各層のノードの数を示す。

nb\_layersとlayer\_sizsをそれぞれ独立して変化させていった。その結果、nb\_layersは
5程度で、layer\_sizesは3000程度でスコアが最大になった。

weight_init_stddevs

This represents the initial value of the weight. It will be adjusted to get better validation score
during the training.


bias_init_consts

The same as weight. This is the initial value of the bias. It will be adjusted from here during the training.

penalty

seisokuka wo okonautokini kyokutannna de-ta no omomi ni taisuru penaruthi. tumari, kore ga ookikereba ookiihodo
gennzaino kaikityokusenn kara ookikuzureru de-ta no eikyouga tiisakunarutoiukotodearu.

https://products.sint.co.jp/aisia/blog/vol1-8

\newpage

\section{機械学習を用いた薬剤活性の予測}

python向けのライブラリであるdeepchemをインストールし、そのチュートリアルのひとつである「Multitask Networks on MUV」をベースにして、薬剤活性の予測を行った。

まず、チュートリアルに入っているハイパーパラメータで50回実行したときの結果は図\ref{fig1}のようになった。


このスコアが更に良くなるように、ハイパーパラメータを調整したい。
そこで、
もとのハイパーパラメータの変数を一つずつ変更し、平均スコアがどのように変化するかを調べた。その結果は、以下のようになった。


ここまで、実験を行ったところで、より適切なハイパーパラメーターを設定できるように、それぞれのパラメーターの意味について調べて以下の表にまとめた。

\begin{table}[htb]
	\begin{tabular}{|l||l|} \hline
		パラメーター名        & 説明 \\ \hline \hline
		momentum             & 
エラー関数の大域的最小値を探そうとする際に、局所解を求めるようになっている。この値が大きければ大きいほど、アルゴリズムの「勢い」のようなものが大きくなり、
ある局所解が求まったとしても、より最適な解を探そうとするようになる。\\ \hline
		batch\_size           & 
一度のイテレーションに用いられるトレーニングサンプルの数。小さすぎると、結果にノイズが乗りやすくなる。一方で、大きすぎると、テストデータが足りなくなる。 \\ \hline
		learning\_rate, decay &
アルゴリズムがデータから学習する際に、新しいデータの取り込みに対する
感度を示す。learning\_rateの値が大きければ大きいほど、新しいデータに強く適応するようになる。
また、データを取り込むにつれてあとから入ってくるデータの重みが小さくなっていくが、decayはその小さくなっていく速さを示す。\\ \hline
		nb\_epoch	     &
トレーニングサンプル間で、何度epochを行うか。epochとは、すべてのトレーニングデータ間で
フォワードパスとバックワードパスを一度行うこと。つまり、各データに対してイテレーションを
行うことである。\\ \hline
		dropout              & 過激適合を避けるために、ノード間のウェイトを均等にする必要がある。そのために、ランダムにレイヤーのノードをオフにする、
dropoutというものが行われるが、これはそれが起きる頻度を示す。\\ \hline
		nb\_layers, layer\_sizes &
ニューラルネットワークの層の数、および各層のノードの数を示す。\\ \hline
		weight\_init\_stddevs  &
重みの初期値を示す。これは、訓練中に調整されていく。\\ \hline
		bias\_init\_consts     &
バイアスの初期値を示す。ウェイトと同じく、これも訓練中に調整されていく。\\ \hline
		penalty              &
正則化を行う際の、極端なデータの重みに対するペナルティーを示す。これが大きいと、極端なデータの重みがより小さくなり、モデルに影響を与えにくくなる。\\ \hline
	\end{tabular}
\end{table}

これらの情報から、ニューラルネットワークのレイヤーの変更に関わらないパラメーターだけを調整し、1レイヤーでは
次のようなパラメーターでスコアの平均をを0.613569まで上げることができた。しかも、十回の思考での分散が0.005626ともともとのパラメーターのときよりも小さくなっているので、安定して6割程度の制度を得た。



\newpage

\section{MUVデータセットについて}

まず、このデータ・セットがなんのためにあるのかについて、論文\cite{thesis}を読んでわかったことを以下で説明する。


現代の薬学の世界では、バーチャールスクリーニングを行うことにより、薬学的に役に立つであろう物質を選び出すことが行われている。
ところが、最近の研究により分子構造に基づいたバーチャルスクリーニング(SBVS)は、バリデーションで用いられるデータセットが結果に大きな影響を及ぼすことがわかった。特に、データ・セット間で分子量・水素結合の数等の「シンプル」な性質が大きく異なっていると、スコアが上がったかのように見えることにこのデータ・セットは注目している。
したがって、ある一定の範囲に収まる「シンプル」な性質を持つものを集めたデータ・セットでなければ、正しいバリデーションとは言えず、バイアスがかかってしまうと考えられる。今回用意されているデータ・セットは、そのようなバイアスがかからないように、集められたものであり、これはMaximum Unbiased Validation(最もバイアスが小さいバリデーション)メソッドを用いたバーチャルスクリーニングを可能にする。


次に、データ・セットに含まれていたタンパク質のうちの一つに注目し、それに対して活性のある化合物と活性のない化合物を挙げたる。

今回注目したタンパク質は、MVUのデータ・セットに含まれる、MUV-600というものである。
MUV-600の構造は以下のとおりになっている。
\cite{MUV-600}

このタンパク質に対し、活性のある化合物として以下のような構造をしているものが例として挙げられる。

一方、活性のない化合物としては以下のようなものが挙げられる。

\begin{thebibliography}{9}
	\bibitem{thesis} Sebastian G. Rohrer and Knut Baumann, "Maximum Unbiased Validation (MUV) Data Sets for Virtual Screening Based on PubChem Bioactivity Data, " (URL: https://pubs.acs.org/doi/10.1021/ci8002649 as of Oct. 21 2018.)
	\bibitem{MUV-600} PubChem MUV-600: \url{https://pubchem.ncbi.nlm.nih.gov/bioassay/600#section=Protocol} Protein Target: \url{https://pubchem.ncbi.nlm.nih.gov/target/protein/NP_004950}
	\bibitem{regularization} 過学習を防ぐ方法 \url{https://products.sint.co.jp/aisia/blog/vol1-8}
\end{thebibliography}
\end{document}
